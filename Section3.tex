\section{Conclusion}

\textbf{En conclusion}, ce projet a été une expérience enrichissante qui a permis de renforcer nos compétences en gestion de projet, même si nous n'avons pas pu terminé toutes les tâches demandées. Travailler en binôme nous a permis de diviser efficacement les tâches et de collaborer de manière complémentaire. Chacun a pu se concentrer sur des aspects spécifiques du projet, ce qui a facilité l’avancement et la qualité du travail.
\newline

Cependant, nous avons rencontré quelques défis en cours de route, notamment la syntaxe LaTeX et la synchronisation de nos travaux LaTeX sur le dépôt en ligne. Nous avons également pris conscience de l'importance de bien communiquer afin d’éviter les erreurs liées à des tâches mal attribuées ou à des malentendus concernant les fonctionnalités attendues.
\newline

Ce projet nous a donné un aperçu de la manière dont les entreprises utilisent Git pour travailler en équipe, tout en nous confrontant à des aspects techniques spécifiques que nous n’avions pas encore maîtrisés, comme les pull requests, les issues, etc. Si nous devions refaire ce projet, nous serions davantage mieux préparés sur les étapes de planification et sur la répartition des tâches afin de maximiser notre efficacité.
\newline

Bien que quelques ajustements puissent être faits pour améliorer l’organisation et la fluidité du travail, ce projet a été une réussite, nous permettant de mieux comprendre l’importance du travail collaboratif et de la gestion de projet.